\ac{SDN} is a new technology which gives the network administrator greater power over his
network. Such control is given through the separation between the control plane and the
data plane, which characterizes an \ac{SDN}. 
%
In this paper it is conceptualized several key points relative to \ac{SDN}, such as \textit{OpenFlow}
and the \ac{SDN} planes. In the following section it is described the networking simulation tool
\textit{Mininet} folowwing by the description of two benchmarks that will be used with the 
objective of data collection for later analisys.
%
Since the hardwares that supports \ac{SDN} are relatively new, their costs is very often
prohibitively high for a small research group or an independent researcher, so that the
use of simulators becomes indispensable for the technological and scientific development in
the field.
%
This paper has as its major objective to do an comparative between the transferering of
medium sized data in a Mininet simulated environment and a traditional networking model 
(i.e.\ coupled Data Plane and Control plane).
%
%
%Assistive Technology (AT) is a knowledge field used to identify resources in order to provide or
%expand skills of people with disabilities, mobility disability or reduced mobility. 
%This work initially presents a bibliography research about people with disabilities, specifically
%people who have Cerebral Palsy (CP) followed by an applied research in order to specify the
%requirements and project for a Extended Alternative Communication software solution. 
%Based on the set of people with CP, this work addresses specifically the people who have limited
%locomotor skills in addition to speech difficulties. 
%The alternative solution software specification presented in this work allows these people to
%communicate with their therapists in order to stimulate their cognition. 
%This paper is organized as follows: context, research results related to people with disabilities;
%AT initiatives and their taxonomies, the specification of the problem that is addressed in the
%work; and specification of the proposal suggested by the author. 
%However, the AT initiatives tend to have some regionalism directed related to local demands. 
%Thus, this work also includes interactions with the Associa��o dos Deficientes de Joinville (ADEJ)
%in order to identify this regional context and the Grupo Assitiva of the Santa Catarina State
%University (UDESC).
