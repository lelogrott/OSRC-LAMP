The use of biometric systems mainly applied to security have been explored through the development of solutions based on detection of human characteristics such as voice signals, fingerprints and faces. The identification of faces and its subsequent recognition is one of the most studied biometric techniques, presenting several methods already proposed based on image processing, but limited by difficulties related to lighting, brightness and facial position, for example. A technique using values of extinction method based on relevant regional extremes of an image in gray levels, is studied to aid in the recognition of faces in frontal position. Tree inicial alternatives are proposed for the use of extinction in this application, the first three are designed to form triangles and measure its internal angles, creating Voronoi diagram and calculate the average gray levels for each area, and isolating the eyes, nose and mouth, to calculate the average levels for each region. Since the three proposals presented unsatisfactory results, a new solution is developed using the coordinates of the $k$ largest extinctions extracted and their normalizations to create simplified images, and applied to PCA for recognition. Furthermore, a second contribution is defined with morphological reconstruction of images by the higher volume extinction, and recognition by PCA technique. Thus, through the first contribution is possible to obtain high compression ratios, better performance and a recognition of $82.35$\% of the input faces, beyond the second contribution, which generates files in a smaller size and has the same rate of recognition that the original images for the PCA algorithm.


