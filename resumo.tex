O \ac{SDN} � uma tecnologia recente que permite ao administrador de redes um maior controle sobre uma rede.
Tal controle � obtido atrav�s da separa��o entre o \textit{Control Plane} e \textit{Data Plane}, o que
caracteriza uma \ac{SDN}.
%
Neste trabalho s�o conceituados diversos pontos chaves relativos ao assunto, tais como \textit{OpenFlow} e os
planos do \ac{SDN}. Em seguida � descrita a ferramenta de simula��o de redes \textit{Mininet} e no fim do
trabalho � descrito dois \textit{benchmarks} com o objetivo de coletar dados para uma an�lise de
desempenho.
%
Tendo em vista que os \textit{hardwares} que suportam \ac{SDN} s�o relativamente recentes, o seu custo
� muitas vezes proibitivo para um pequeno grupo de pesquisa ou um pesquisador independente, de tal forma
que o uso de simuladores se torna indispens�vel para o desenvolvimento cientifico e tecnol�gico na
�rea.
%
Este trabalho tem como objetivo realizar um comparativo entre a transfer�ncia de dados de tamanho m�dio
dentro de um ambiente simulado no \textit{Mininet} e um ambiente utilizando o modelo tradicional de rede
(\textit{Data Plane} e \textit{Control Plane} acoplados).
%
