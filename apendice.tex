\appendix

\chapter{Exemplo num{\'e}rico da primeira contribui{\c{c}}{\~a}o}

Considerando como entrada ao algoritmo uma imagem em n{\'i}veis de cinza de dimens{\~o}es $1 \times 20$, representada na figura \ref{fig:entrada_exemplo_numerico}, observa-se a exist{\^e}ncia de dois m{\'a}ximos regionais: $50$ e $43$. Por meio destes m{\'a}ximos, {\'e} poss{\'i}vel encontrar suas $k$ maiores extin{\c c}{\~o}es. No exemplo adotado, ser{\'a} considerada a maior extin{\c c}{\~a}o de cada tipo de atributo crescente e ser{\~a}o apresentados os c{\'a}lculos baseados no atributo altura.

\begin{figure}[ht]
	\centering
	\caption{Imagem de entrada do exemplo.}
	\includegraphics[scale=1.0]{imagens/exemplo_numerico.png} \\
	Fonte: Imagem gerada pela autora.
    \label{fig:entrada_exemplo_numerico}
\end{figure}

A partir da imagem de entrada, portanto, {\'e} necess{\'a}rio encontrar o m{\'a}ximo regional com a maior extin{\c c}{\~a}o de altura. Analisando a figura \ref{fig:entrada_exemplo_numerico_extincao}, {\'e} determinada a maior extin{\c c}{\~a}o de altura no m{\'a}ximo $50$, este utilizado nos c{\'a}lculos para gera{\c c}{\~a}o da imagem simplificada. Assim, a partir das coordenadas deste m{\'a}ximo regional {\'e} calculado seu {\'i}ndice representativo a partir do intervalo de normaliza{\c c}{\~a}o escolhido. Este c{\'a}lculo {\'e} efetuado por meio de duas f{\'o}rmulas:

\begin{figure}[ht]
	\centering
	\caption{Extin{\c c}{\~o}es de altura.}
	\includegraphics[scale=1.0]{imagens/exemplo_numerico_extincao.png} \\
	Fonte: Imagem gerada pela autora.
    \label{fig:entrada_exemplo_numerico_extincao}
\end{figure}

\begin{equation}
	I = x_{c} \times nc + y_{c}
\end{equation}

\begin{equation}
	In = i \ (nl \times nc) \times Indice
\end{equation}

onde $x_{c}$ e $y_{c}$ representam, respectivamente, a linha e a coluna da coordenada do m{\'a}ximo de maior altura, $nl$ e $nc$ o n{\'u}mero de linhas e colunas da imagem de entrada, $i$ o n{\'u}mero de unidades do intervalo de normaliza{\c c}{\~a}o, e $I$ e $In$ os valores dos {\'i}ndices antes e ap{\'o}s a normaliza{\c c}{\~a}o, respectivamente. Assim, a partir do valor da coordenada ($0$, $4$) do m{\'a}ximo $50$ e considerando o intervalo de normaliza{\c c}{\~a}o de $0$ a $255$, {\'e} determinado um {\'i}ndice normalizado que o representar{\'a}:

\begin{equation}
	Indice = 0 \times 20 + 4 = 4
\end{equation}

\begin{equation}
	Indice\_normalizado = 255 \ (1 \times 20) \times 4 = 51
\end{equation}

Desta forma, $51$ {\'e} o valor do {\'i}ndice normalizado da maior extin{\c c}{\~a}o de altura da imagem de entrada. Este n{\'u}mero corresponde {\`a}primeira linha ({\'i}ndices do atributo altura) e coluna da imagem simplificada. O c{\'a}lculo dos {\'i}ndices normalizados {\'e} repetido para todos os atributos de valores de extin{\c c}{\~a}o, formando assim, para este exemplo, a imagem simplificada:

\begin{center}
$\left[
	\begin{array}{c}
		51 \\
		ind\_area \\
		ind\_volume \\
		ind\_altura\_caixa \\
		ind\_largura\_caixa \\
		ind\_nro\_descendentes \\
		ind\_altura\_topologica 
	\end{array} 
\right]$
\end{center}

sendo $ind\_area$, $ind\_volume$, $ind\_altura\_caixa$, $ind\_largura\_caixa$, $ind\_nro\_de$ $scendentes$ e $ind\_altura\_topologica$ os {\'i}ndices correspondentes {\`a} maior extin{\c c}{\~a}o de {\'a}rea, volume, altura e largura da caixa envolvente, n{\'u}mero de descendentes e altura topol{\'o}gica da sub-{\'a}rvore, respectivamente. Ap{\'o}s a gera{\c c}{\~a}o das imagens simplificadas para entradas e treinamentos, estas s{\~a}o aplicadas ao algoritmo do PCA para o reconhecimento.
