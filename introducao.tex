\chapter{Introdu��o}
\label{cap:introducao}

Dentre os servidores \textit{web} o \ac{LAMP} ganha 
cada vez mais espa�o no mercado por oferecer solu��es \text{open source}, ou seja, 
sem custo e adapt�veis � cada aplica��o.
Por�m, um dos grandes problemas que afetam servidores utilizando \ac{LAMP} � a
facilidade com que atacantes conseguem invadir e muitas vezes realizar extra��o de informa��es privilegiadas,
o que deixa pessoas um pouco receosas ao utiliz�-lo. Entretanto, neste trabalho apresentam-se testes de invas�o em servidores \ac{LAMP}
(realizados em ambientes controlados e pr�prios para isso), mostrando como � feita e comprovando a possibilidade de invas�o destes
servidores.

No cap�tulo \ref{cap:introducao} � explicado o funcionamento
do sistema \ac{LAMP}, a integra��o dos componentes, o por qu� da
realiza��o de testes de invas�o e quais s�o as normas e 
recomenda��es para a an�lise de vulnerabilidade. Em \ref{lampkali}
decorre-se sobre as vulnerabilidades do ambiente \ac{LAMP} e quais
os m�todos de explora��o utilizados para identificar e replicar
os principais ataques. Detalhes sobre o ambiente de testes bem como
dos ataques realizados, sua an�lise e solu��es relacionadas
s�o expostos em \ref{ataques}.

